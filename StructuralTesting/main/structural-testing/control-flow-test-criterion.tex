\begin{frame}[parent={cmap:structural-software-testing},hasnext=true,hasprev=false]
\frametitle{Control flow test criterion}
\label{concept:control-flow-test-criterion}

\begin{block:concept}{Definition}
\begin{itemize}
	\item Control flow test criterion employs only characteristics related
	to the control structure of the program to determine the set of test
	requirements.

	\item Control flow test criterion identifies the execution paths
	inside a module and then creates and executes test cases to cover those
	paths.
\end{itemize}
\end{block:concept}

\begin{block:fact}{Control flow test criteria}
\begin{itemize}
	\item All-Nodes.
	\item All-Edges.
	\item All-Paths.
\end{itemize}
\end{block:fact}


%\begin{block:fact}{Control flow test criteria and JaBUTi}
%\begin{itemize}
%	\item Regarding control-flow, JaBUTi supports the following test criteria:
%	All-Nodes and All-Edges.
%\end{itemize}
%\end{block:fact}
\end{frame}


\begin{frame}
\frametitle{Control flow test criterion}

\begin{block:fact}{Limitations}
\begin{itemize}
	\item In control flow test criterion, the number of test requirements
	can be huge and thus untestable within a reasonable amount of time:
	\begin{itemize}
		\item Every decision doubles the number of paths; every loop multiplies
		the paths by the number of iterations through the loop.
	\end{itemize}

	\item Paths called for in the specification may simply be missing in the
	module.

    \item Defects may exist in processing statements within the module even
	through the control flow itself is correct.

    \item The module may execute correctly for almost all data values but fail
	for a few.
\end{itemize}
\end{block:fact}
\end{frame}



\begin{frame}
\label{concept:all-nodes-criterion}
\label{concept:all-nodes}
\frametitle{Control flow test criterion}
\framesubtitle{All-Nodes}

\begin{block:concept}{Definition}
All-Nodes requires a test set that exercises at least once each node of the
control flow graph, which is equivalent to executing all code blocks of a
program at least once.
\end{block:concept}

\hfill
\refie{example:all-nodes}{\beamerbutton{Example: All-nodes example}}
\end{frame}


\begin{frame}
\frametitle{Control flow test criterion}
\framesubtitle{All-Nodes}

\begin{block:fact}{Limitations}
\begin{itemize}
	\item Even though the All-Nodes criterion is the simplest structural test
	criterion, it may be difficult to satisfy in practice:
	\begin{itemize}
		\item Often programs have code that is executed only in exceptional
		circumstances-low memory, full disk, unreadable files, lost
		connections, etc.

		\item Testers may find it difficult or even impossible to simulate
		these circumstances and thus code that deals with these problems will
		remain untested.
	\end{itemize}
\end{itemize}
\end{block:fact}
\end{frame}



\begin{frame}
\label{concept:all-edges-criterion}
\label{concept:all-edges}
\frametitle{Control flow test criterion}
\framesubtitle{All-Edges}


\begin{block:concept}{Definition}
All-Edges requires a test set which traverses at least once each edge
of the control flow graph, i.e., the test set must ensure that each
conditional statement assumes true and false values at least once.
\end{block:concept}
\end{frame}


\begin{frame}[hasnext=false,hasprev=true]
\frametitle{Control flow testing criterion}
\framesubtitle{All-Paths}
\label{concept:all-paths-criterion}
\label{concept:all-paths}

\begin{block:concept}{Definition}
All-Paths requires a test set that executes all possible paths of the control
flow graph.
\end{block:concept}

\begin{block:fact}{Limitations}
\begin{itemize}
	\item For program units without loops, the test requirements for the
	All-Paths criterion is generally small enough so that a test case can
	actually be constructed for each path.

	\item For program units with loops, the test requirements for the All-Paths
	criterion can be enormous and thus pose an intractable test problem.
\end{itemize}
\end{block:fact}
\end{frame}
