\begin{frame}[parent={cmap:structural-software-testing},hasnext=true,hasprev=true]
\frametitle{Path}
\label{concept:path}

\begin{block:concept}{Informal definition}
A path is a sequence of statements.
\end{block:concept}

\begin{block:concept}{Definition}
A path is a finite sequence of nodes $(n_1, n_2, . . . , nk)$,
$k \geqslant 2$, so that there is an edge from $n_i$ to $n_i + 1$ for
$i = 1, 2, ... , k - 1$.
\end{block:concept}
\end{frame}


\begin{frame}
\frametitle{Path}
\framesubtitle{Executable and infeasible path}
\label{concept:infeasible-path}
\label{concept:missing-path}

\begin{block:concept}{Executable path}
An executable path is a path for which it exists an input data that can
execute it.
\end{block:concept}

\begin{block:concept}{Infeasible}
An infeasible path is a path that, for any input value, cannot be executed.
\end{block:concept}

\begin{block:fact}{Limitations and implications}
\begin{itemize}
	\item It is impossible to determine, automatically, infeasible paths.

	\item Any complete path that includes an infeasible path is an
	infeasible path.
\end{itemize}
\end{block:fact}

\hfill
\refie{example:identifier-infeasible-path}{\beamerbutton{Example: Infeasible path example for Identifier}}
\end{frame}



\begin{frame}
\frametitle{Path}
\framesubtitle{Definition-clear path}
\label{concept:definition-clear-path}

\begin{block:concept}{Informal definition}
A definition-clear path is a path which no other variable definition is made but
on the entry node.
\end{block:concept}

\begin{block:concept}{Definition}
A definition-clear path with respect to a variable $x$ is a path between two
nodes $A$ and $B$, being $x$ defined in $A$, with an use in $B$ and with no
other definition of $x$ in the other nodes present in the path between $A$ and
$B$.
\end{block:concept}

\hfill
\refie{example:identifier-def-clear-path}{\beamerbutton{Example: Definition-clear path for Identifier}}
\end{frame}