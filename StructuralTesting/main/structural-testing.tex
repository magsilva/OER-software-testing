\begin{frame}[parent={cmap:software-testing}, hasprev=false, hasnext=true]
\frametitle{Structural testing}
\label{cmap:structural-software-testing}
\label{cmap:structural-testing}

\insertcmap{Courses-SoftwareTesting-StructuralTesting}
\end{frame}



\begin{frame}[parent={cmap:structural-software-testing},hasnext=true,hasprev=true]
\frametitle{Structural testing}
\label{concept:structural-testing}

\begin{block:concept}{Definition}
Structural testing is a technique in which testing is based on the
internal paths, structure, and implementation of the software under test.
\end{block:concept}

\begin{block:fact}{White-box testing}
As structural testing must see the inner details of the software, it is also
known as white box testing.
\end{block:fact}

\begin{block:fact}{Why is structural testing important?}
\begin{itemize}
	\item Structural testing has efficacy on determine logical or programming
	faults in the program under testing, specially at the unit level.
\end{itemize}
\end{block:fact}
\end{frame}


\begin{frame}
\frametitle{Structural testing}

\begin{block:concept}{Limitations}
\begin{itemize}
	\item Structural testing requires detailed programming skills.
	\begin{itemize}
		\item Structural testing requires tester intervention in order to
		determine infeasible paths.
	\end{itemize}

	\item The number of execution paths may be so large that they cannot all be
	tested.

	\item The test cases chosen may not detect data sensitivity errors.

	\item Structural testing assumes that control flow is correct (or very
	close to correct). Since the tests are based on the existing paths, nonexistent
	paths cannot be usually discovered through structural testing.
\end{itemize}
\end{block:concept}
\end{frame}


\begin{frame}
\frametitle{Structural testing}

\begin{block:fact}{When can I use structural testing?}
\begin{itemize}
	\item Structural testing can be applied at unit, integration, and system
	testing phases.
\end{itemize}
\end{block:fact}


\begin{block:fact}{Structural testing and test phases}
\begin{itemize}
	\item Structural testing, when applied at the unit testing phase,
	involves paths that are within a module.

	\item Structural testing, when applied at the integration testing phase,
	involves paths that are between modules within subsystems and paths
	between subsystems within systems.

	\item Structural testing, when applied at the system testing phase,
	involves paths that are between entire systems.
\end{itemize}
\end{block:fact}
\end{frame}


\begin{frame}[hasprev=true, hasnext=false]
\frametitle{Structural testing}

\begin{block:procedure}{Test activities}
\begin{enumerate}
	\item The implementation of the program under testing is analyzed.
	\item Paths through the program under testing are identified.
	\item Inputs are chosen to cause the program under testing to execute
	selected paths. This is called path sensibilization.
	\item Expected outputs for those inputs are determined.
	\item The test cases are run.
	\item Actual outputs are compared with the expected outputs, verifying if
	the the actual output is correct.
\end{enumerate}
\end{block:procedure}
\end{frame}