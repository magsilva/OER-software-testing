\begin{frame}[parent={concept:junit}, hasprev=false, hasnext=true]
\frametitle{JUnit}
\framesubtitle{Installation (command line)}
\label{procedure:junit:installation:cmdline}


\begin{block:fact}{Requirements}
\begin{itemize}
	\item JUnit requires the Java SDK 1.5 or newer.
\end{itemize}
\end{block:fact}

\begin{block:procedure}{}
\begin{enumerate}
	\item Download JUnit at \url{https://github.com/junit-team/junit4/wiki/Download-and-Install}.
	\begin{itemize}
		\item Current version is 4.12.

		\item The application is distributed as two JAR files:
		\begin{itemize}
			\item \srccode{junit.jar}: main JUnit library
			\item \srccode{hamcrest-core}: library of matchers (optional, only required for \srccode{assertThat)}
		\end{itemize}
	\end{itemize}
\end{enumerate}
\end{block:procedure}
\end{frame}


\begin{frame}[fragile, hasprev=true, hasnext=true]
\frametitle{JUnit}
\framesubtitle{Installation (command line)}

\begin{block:fact}{Classpath configuration}
	\begin{itemize}
		\item You can add the library to the CLASSPATH environment variable.
\begin{lstlisting}
Unix:
export CLASSPATH=/opt/junit/junit.jar:
  /opt/junit/hamcrest-core.jar:$CLASSPATH
		
Windows:
set CLASSPATH=C:\junit\junit.jar;
  C:\junit\hamcrest-core.jar;%CLASSPATH%
\end{lstlisting}
		
		\item You can use the -cp option when running the tests. This is the
		recommended option!
\begin{lstlisting}
java -cp /opt/junit/junit.jar:/opt/junit/hamcrest-core.jar  <program>
\end{lstlisting}
	\end{itemize}
\end{block:fact}
\end{frame}



\begin{frame}
\frametitle{JUnit}
\framesubtitle{Installation (Eclipse)}
\label{procedure:junit:installation:eclipse}


\begin{block:fact}{Requirements}
	\begin{itemize}
		\item Any Eclipse version
	\end{itemize}
\end{block:fact}

\begin{block:procedure}{}
	For each project you want to use JUnit, proceed as follows:
	\begin{enumerate}
		\item Access the project's properties.
		\item Select \srccode{Java Build Path} tab on the left.
		\item Select \srccode{Libraries} tab on the right.
		\item Select \srccode{Add Library} button on the right of \srccode{Libraries} tab.
		\item Select \srccode{Junit}.
		\item Proceed to the next window by pressing the \srccode{Next button}.
		\item Check if JUnit version is \srccode{JUnit 4}.
		\item Press \srccode{Finish} button.
		\item Press \srccode{Apply} button.
		\item Press \srccode{Ok} button.
	\end{enumerate}
\end{block:procedure}
\end{frame}


\begin{frame}[hasprev=true, hasnext=false]
\frametitle{JUnit}
\framesubtitle{Shake down}
\label{procedure:junit:shakedown}


\begin{block:fact}{Is it working?}
\begin{itemize}
	\item To check whether JUnit was correctly installed, you can run the JUnit
	test suite.
	\begin{itemize}
		\item The class with all the test cases for JUnit is
		\srccode{org.junit.tests.AllTests}.

		\item This class is located at the root of JUnit installation directory.
	\end{itemize}

	\item Or you may create your own test set! Check the example below.
\end{itemize}
\end{block:fact}


\hfill
\refie{example:junit-shakedown}{\beamerbutton{Example: JUnit shakedown}}
\end{frame}
