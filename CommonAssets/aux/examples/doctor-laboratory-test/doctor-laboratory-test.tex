\begin{frame}[hasprev=false, hasnext=true]
\frametitle{Test case success and failure analogy}
\framesubtitle{Doctor and laboratory tests}
\label{example:doctor-laboratory-test}

\begin{block}{Sick person visiting a doctor}
\begin{itemize}
	\item Consider the analogy of a person visiting a doctor because of an
	overall feeling of malaise.

	\item If the doctor runs some laboratory tests that do not locate the
	problem, we do not call the laboratory test ``successful''.
	\begin{itemize}
		\item They were unsuccessful tests in that the patient's net worth has
		been reduced by the expensive laboratory fees, and the patient is still
		ill.
	\end{itemize}

	\item However, if a laboratory test determines that the patient has a
	peptic ulcer, the test is successful.
	\begin{itemize}
		\item The doctor can now begin the appropriate treatment.
	\end{itemize}
\end{itemize}
\end{block}

\end{frame}


\begin{frame}[hasprev=true, hasnext=false]
	\frametitle{Test case success and failure analogy}
	\framesubtitle{Doctor and laboratory tests}

	\begin{block}{Analogy}
		\begin{itemize}
			\item The sick person should be imagined as the software.
			
			\item The laboratory tests are the test cases designed for the software.
			
			\item Peptic ulcer is the failure found by the test.
		\end{itemize}
	\end{block}
\end{frame}