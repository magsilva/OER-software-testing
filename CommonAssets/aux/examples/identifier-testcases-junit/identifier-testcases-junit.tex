\begin{frame}[hasprev=false, hasnext=false]
\frametitle{Identifier}
\label{example:identifier-testcases-junit}

\begin{itemize}
	\item The program determines if a given identifier is valid or not in a
	variant of Pascal language, called Silly Pascal.

	\item A valid identifier must begin with a letter and must contain only
	letter or digits.

	\item Moreover, it must have at least one character and no more than six
	characters.
\end{itemize}
\end{frame}



\begin{frame}[fragile, hasprev=true, hasnext=true]
\frametitle{Identifier}
\framesubtitle{Test set fixture}


\begin{lstlisting}
package identifier;

import org.junit.Test;
import org.junit.Assert;

public abstract class IdentifierTestSet
{
	protected Identifier id;

	@Before
	public void setUp() {
		id = new Identifier();
	}
}
\end{lstlisting}
\end{frame}




\begin{frame}[fragile]
\frametitle{Identifier}
\framesubtitle{Test set 1}

\begin{lstlisting}
package identifier;

import org.junit.*

public class IdentifierTestSet1 extends IdentifierTestSet
{
  @Test
  public void validate1() {
    boolean result = id.validateIdentifier("Abcd5");
    Assert.assertEquals(true, result);
  }

  @Test
  public void validate2() {
    boolean result = id.validateIdentifier("x12345");
    Assert.assertEquals(true, result);
  }
}
\end{lstlisting}
\end{frame}




\begin{frame}[fragile]
\frametitle{Identifier}
\framesubtitle{Test set 2}

\begin{lstlisting}
package identifier;

import org.junit.*

public class IdentifierTestSet2 extends IdentifierTestSet
{
	@Test
	public void validate3() {
		boolean result = id.validateIdentifier("&123");
		Assert.assertFalse(result);
	}

	@Test
	public void validate4() {
		boolean result = id.validateIdentifier("Inv@lido");
		Assert.assertFalse(result);
	}
}
\end{lstlisting}
\end{frame}



\begin{frame}[fragile]
\frametitle{Identifier}
\framesubtitle{Test set 3}

\begin{lstlisting}
package identifier;

import org.junit.*;

public class IdentifierTestSet3  extends IdentifierTestSet
{
	@Test
	public void validate5() {
		Assert.assertNotNull(id);
	}

	@Test(expected=IndexOutOfBoundsException.class)
	public void stringException() {
		String str = new String("JUnit Example");
		str.substring(30);
	}
}
\end{lstlisting}
\end{frame}


\begin{frame}[fragile]
\frametitle{Identifier}
\framesubtitle{Test set 4}

\begin{lstlisting}
package identifier;

import org.junit.*;

public class IdentifierTestSet4  extends IdentifierTestSet
{
	@Test(timeout=2000)
	public void looping() {
		boolean result = id.validateIdentifier("Abcd5");
		Assert.assertEquals(true, result);
	}

	@Ignore("Out of the program scope")
	@Test(expected=IndexOutOfBoundsException.class)
		public void stringException2() {
		String str = new String("JUnit Example");
		str.substring(30);
	}
}
\end{lstlisting}
\end{frame}



\begin{frame}[fragile, hasprev=true, hasnext=false]
\frametitle{Identifier}
\framesubtitle{Test suite}

\begin{lstlisting}
package identifier;

import org.junit.runner.RunWith;
import org.junit.runners.Suite;

@RunWith(Suite.class)
@Suite.SuiteClasses({
	IdentifierTestSet1.class,
    IdentifierTestSet2.class
    IdentifierTestSet3.class
    IdentifierTestSet4.class
})
public class AllTests
{
}
\end{lstlisting}
\end{frame}