\begin{frame}[hasprev=false, hasnext=true, fragile]
\label{example:blech-exhaustive-testing}
\frametitle{Exhaustive testing}
\framesubtitle{Blech example}

\begin{block:ie}{}
\begin{itemize}
	\item Consider the function \srccode{blech}, implemented as follows:
\begin{lstlisting}
int blech(int j) {
    j = j - 1; // should be j = j + 1;
    j = j / 30000;
    return j;
}
\end{lstlisting}

	\item Input domain:
	\begin{itemize}
		\item Consider an integer type of 16 bits (2 bytes).

		\item The lowest possible input value is -32,768 and the highest is
		32,767.

		\item Thus there are 65,536 possible inputs into this small software.
	\end{itemize}

	\item Which test cases can detect the fault?
\end{itemize}
\end{block:ie}
\end{frame}



\begin{frame}[hasprev=true, hasnext=false, fragile]
\frametitle{Exhaustive testing}
\framesubtitle{Blech example}

\begin{block:ie}{}
\begin{itemize}
	\item Only four out of the possible 65,536 input values will find this
	fault:
\end{itemize}

{
	\centering
	\small
    \begin{tabular}{|l|l|l|}
	\hline
	Test Cases Input(j)	& Expected Output	& Actual Result \\
	\hline
	-30000 & 0 & -1\\
	\hline
	-29999 & 0 & -1\\
	\hline
	30000 & 1 & 0\\
	\hline
	29999 & 1 & 0\\
	\hline
    \end{tabular}
}

\begin{lstlisting}
int blech(int j) {
    j = j - 1; // should be j = j + 1;
    j = j / 30000;
    return j;
}
\end{lstlisting}
\end{block:ie}
\end{frame}

